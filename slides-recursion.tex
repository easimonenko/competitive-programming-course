% slides-recursion.tex
% Competitive Programming Course (in Russian)
% Author: Evgeny Simonenko <easimonenko@mail.ru>
% License: CC BY-ND 4.0

\documentclass[11pt]{beamer}

\usepackage[utf8x]{inputenc}
\usepackage[OT1]{fontenc}
\usepackage[english, russian]{babel}
\usepackage{graphicx}

\usetheme{Boadilla}

\usepackage{listings}
\lstset{fontadjust=true}

\usepackage{alltt}

\begin{document}

\author{Симоненко Е.А. \\ \texttt{easimonenko@mail.ru}}
\title{Спортивное программирование}
\subtitle{Рекурсия и применение её к решению задач}
\date{2018}
\setbeamertemplate{navigation symbols}{}

\begin{frame}
\titlepage
\end{frame}

\begin{frame}
\frametitle{Содержание}
\tableofcontents
\end{frame}

\section{Что такое рекурсия}

\begin{frame}[fragile]
\frametitle{Факториал}
\[n! = 1 \cdot 2 \cdot ... \cdot (n - 1) \cdot n\]
\[n! = n \cdot \underbrace{(n - 1) \cdot (n - 2) \cdot ... \cdot 1}_{(n - 1)!}\]
\[ \left\{
\begin{array}{l}
n! = n \cdot (n - 1)! \\
1! = 1
\end{array}
\right. \]
\begin{lstlisting}[frame=single,language=c]
int factorial(int n) {
    return n == 1 ? 1 : n * factorial(n - 1);
}
\end{lstlisting}
\end{frame}

\section{Примеры}

\begin{frame}[fragile]
\frametitle{Числа Фибоначчи}
\[\left\{
\begin{array}{l}
F_n = F_{n - 1} + F_{n - 2} \\
F_1 = 1, F_2 = 1
\end{array}
\right.\]
\begin{lstlisting}[frame=single,language=c]
int fibonacci(int n) {
    return n == 1 || n == 2
        ? 1
        : fibonacci(n - 1) + fibonacci(n - 2);
}
\end{lstlisting}
\end{frame}

\begin{frame}[fragile]
\frametitle{Бинарное возведение в степень}
\[\left\{
\begin{array}{l}
x^n = {x^{n / 2}}^2, \text{если} n - \text{чётное} \\
x^n = x \cdot x^{n - 1}, \text{если} n - \text{нечётное}
\end{array}
\right.\]
\begin{lstlisting}[frame=single,language=c]
int binpow(int x, int n) {
    if (n == 1) {
        return x;
    } else if (n % 2 == 0) {
        int h = binpow(x, n / 2);
        return h * h;
    } else {
        return x * binpow(x, n - 1);
    }
}
\end{lstlisting}
\end{frame}

\begin{frame}
\frametitle{Острова}
\begin{figure}[h]
\center
\begin{alltt}
.............................. \\
.*..*..****..*.....*.....****. \\
.*..*..*.....*.....*.....*..*. \\
.****..***...*.....*.....*..*. \\
.*..*..*.....*.....*.....*..*. \\
.*..*..****..****..****..****. \\
..............................
\end{alltt}

\end{figure}
\end{frame}

\section{Упражнения}

\begin{frame}
\frametitle{Упражнения на применение рекурсии}
\begin{itemize}
	\item \href{https://codeforces.com/problemsets/acmsguru/problem/99999/461}{Wiki Lists}
	\item \href{https://codeforces.com/contest/60/problem/B}{Время сериала!}
\end{itemize}
\end{frame}

\section{Библиография}

\begin{frame}
\frametitle{Библиография}
\begin{thebibliography}{1}
  \bibitem[Порублёв, Ставровский]{PorublevStavrovsky} Порублев И.Н., Ставровский А.Б. Алгоритмы и программы.
  Решение олимпиадных задач. -- М.: Вильямс, 2007. -- 480 с.
  \bibitem[Меньшиков]{Menshikov} Меньшиков Ф.В. Олимпиадные задачи по программированию. -- СПб.: Питер,
  2007. -- 315 с.
  \bibitem[Мозговой]{Mozgovoy} Мозговой М.В. Занимательное программирование. -- СПб.: Питер, 2005. -- 208 с.
  \bibitem[Грэхем, Кнут, Паташник]{} Грэхем Р., Кнут Д., Паташник О. Конкретная математика. Основания информатики: Пер. 
  с англ. -- 3-е изд. -- М.: Мир; БИНОМ. Лаборатория знаний, 2009. -- 703 с.
  \bibitem[Хаггарти]{} Хаггарти Р. Дискретная математика для программистов. – 2-е изд. – М.: Техносфера, 2005. – 400 с.
\end{thebibliography}
\end{frame}

\section{Ссылки}

\begin{frame}
\frametitle{Ссылки}
\begin{itemize}
	\item \url{https://codeforces.com/}
	\item \url{http://acm.timus.ru/}
	\item \url{https://github.com/easimonenko/competitive-programming-course}
\end{itemize}
\end{frame}

\end{document}
