% slides-combinatorics.tex
% Competitive Programming Course (in Russian)
% Author: Evgeny Simonenko <easimonenko@mail.ru>
% License: CC BY-ND 4.0

\documentclass[11pt]{beamer}

\usepackage[utf8x]{inputenc}
\usepackage[OT1]{fontenc}
\usepackage[english, russian]{babel}
\usepackage{graphicx}

\usetheme{Boadilla}

\usepackage{listings}
\lstset{fontadjust=true}

\usepackage{alltt}

\begin{document}

\author{Симоненко Е.А. \\ \texttt{easimonenko@mail.ru}}
\title{Спортивное программирование}
\subtitle{Базовые методы комбинаторной теории}
\date{2018}
\setbeamertemplate{navigation symbols}{}

\begin{frame}
\titlepage
\end{frame}

\begin{frame}
\frametitle{Содержание}
\tableofcontents
\end{frame}

\section{Основные определения}

\begin{frame}[fragile]
\frametitle{Конечные множества}
...
\end{frame}

\begin{frame}[fragile]
\frametitle{Операции над множествами}
...
\end{frame}

\begin{frame}[fragile]
\frametitle{Метод включений и исключений}
...
\end{frame}

\section{Комбинации без повторений}

\begin{frame}[fragile]
\frametitle{}
...
\end{frame}

\section{Комбинации с повторениями}

\begin{frame}[fragile]
\frametitle{}
...
\end{frame}

\section{Примеры}

\begin{frame}[fragile]
\frametitle{Треугольник Паскаля}
...
\end{frame}

\begin{frame}[fragile]
\frametitle{Генерация всех подмножеств}
\[
\begin{array}{l}
(0,0,0) \leftrightarrow \emptyset \\
(0,0,1) \leftrightarrow \{c\} \\
(0,1,0) \leftrightarrow \{b\} \\
(0,1,1) \leftrightarrow \{b,c\} \\
(1,0,0) \leftrightarrow \{a\} \\
(1,0,1) \leftrightarrow \{a,c\} \\
(1,1,0) \leftrightarrow \{a,b\} \\
(1,1,1) \leftrightarrow \{a,b,c\}
\end{array}
\]
\end{frame}

\begin{frame}[fragile]
\frametitle{Код Грея}
\[
\begin{array}{l}
(0,0,0) \leftrightarrow (0,0,0) \leftrightarrow \emptyset \\
(0,0,1) \leftrightarrow (0,0,1) \leftrightarrow \{c\} \\
(0,1,0) \leftrightarrow (0,1,1) \leftrightarrow \{b,c\} \\
(0,1,1) \leftrightarrow (0,1,0) \leftrightarrow \{b\} \\
(1,0,0) \leftrightarrow (1,1,0) \leftrightarrow \{a,b\} \\
(1,0,1) \leftrightarrow (1,1,1) \leftrightarrow \{a,b,c\} \\
(1,1,0) \leftrightarrow (1,0,1) \leftrightarrow \{a,c\} \\
(1,1,1) \leftrightarrow (1,0,0) \leftrightarrow \{a\}
\end{array}
\]
\begin{lstlisting}[language=c]
unsigned int grayencode(unsigned int g) {
    return g ^ (g >> 1);
}
\end{lstlisting}
\end{frame}

\section{Упражнения}

\begin{frame}
\frametitle{Упражнения на применение комбинаторных методов}
\begin{itemize}
	\item 
	\item 
\end{itemize}
\end{frame}

\section{Библиография}

\begin{frame}
\frametitle{Библиография}
\begin{thebibliography}{1}
  \bibitem[Порублёв, Ставровский]{PorublevStavrovsky} Порублев И.Н., Ставровский А.Б. Алгоритмы и программы.
  Решение олимпиадных задач. -- М.: Вильямс, 2007. -- 480 с.
  \bibitem[Меньшиков]{Menshikov} Меньшиков Ф.В. Олимпиадные задачи по программированию. -- СПб.: Питер,
  2007. -- 315 с.
  \bibitem[Мозговой]{Mozgovoy} Мозговой М.В. Занимательное программирование. -- СПб.: Питер, 2005. -- 208 с.
  \bibitem[Грэхем, Кнут, Паташник]{} Грэхем Р., Кнут Д., Паташник О. Конкретная математика. Основания информатики: Пер. 
  с англ. -- 3-е изд. -- М.: Мир; БИНОМ. Лаборатория знаний, 2009. -- 703 с.
  \bibitem[Ландо]{} Ландо С.К. Введение в дискретную математику. -- М.: МЦНМО, 2012. -- 265 с.
\end{thebibliography}
\end{frame}

\section{Ссылки}

\begin{frame}
\frametitle{Ссылки}
\begin{itemize}
	\item \url{https://codeforces.com/}
	\item \url{http://acm.timus.ru/}
	\item \url{https://github.com/easimonenko/competitive-programming-course}
\end{itemize}
\end{frame}

\end{document}
